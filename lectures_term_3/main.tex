		\documentclass[10pt, oneside, a4paper]{book}
\usepackage{ifpdf}
\usepackage[colorlinks,bookmarksopen]{hyperref}
\usepackage[T2A]{fontenc}
\usepackage[utf8]{inputenc}
\usepackage[english, russian]{babel}
\usepackage{amsthm,amsfonts,amsmath, amssymb}
\usepackage{euscript,eufrak}%an comment
\DeclareMathOperator{\s}{\EuScript{S}}
\DeclareMathOperator{\nat}{\mathbb{N}}
\DeclareMathOperator{\real}{\mathbb{R}}
\DeclareMathOperator{\ssum}{\sum_{k=1}^{\infty}}
\DeclareMathOperator{\ssumn}{\sum_{k=1}^{n}}

\theoremstyle{plain}% default
\newtheorem{thm}{Теорема}[chapter]
\newtheorem{lem}[thm]{Лемма}
\newtheorem*{cor}{Следствие}
\newtheorem{st}{Утверждение}[chapter]
\newtheorem*{sts}{Утверждение}
%\newtheorem*{proof}{Доказательство}
%\renewcommand{\bfdefault}{b}

\theoremstyle{definition}
\newtheorem{deff}{Определение}[chapter]
\theoremstyle{remark}
\newtheorem*{note}{Замечание}

\begin{document}
\begin{titlepage}
\author{hayer, the typemaster; dr\_droll corrections}
\title{Лекции\\ По математическому анализу Т.П. Лукашенко}
%\date{}
\maketitle
\end{titlepage}
\setcounter{part}{2}
\part{	Лекции третьего семестра}
\setcounter{chapter}{0}
\chapter{}
\begin{deff}
Пусть $ \{a_k \}_{k=n}^{\infty} $ --- последовательность, занумерованная целыми числами начиная с $ n $ и далее по возростанию.\\
Тогда выражение вида $a_m+a_{m+1}+a_{m+2}+ \cdots = \sum_{k=m}^{\infty} a_k$ называется бесконечным рядом.\\
\end{deff}
Изменением нумерации общий случай можно свести к случаю $n=1$ или $n=0$.
Тоже можно получить при $n>1$ добавлением нулевых членов или заменой начальных членов их суммой в случае $n<0$.
\begin{deff}
$ \s_{N} = \sum_{k=m}^{N} $ --- частичная сумма с номером N.\\
$ \s_{N} = 0 $, если $ N < m $.\\
Если существует предел:
$$ \lim_{N \to \infty} {\s_N} = \s $$
то его называют суммой ряда $\s$. Если $a_k$ --- действительные числа, то $\s$ действительное число или $\pm \infty$.
Ряд называется сходящимся, если его сумма конечна. 
Если это не так, то ряд называют расходящимся.
\end{deff}
\begin{st}Критерий Коши.\\
Числовой ряд сходится тогда и только тогда, когда:
$$ \forall{\epsilon\!>\! 0} \quad \exists N \quad \forall m \! >\! N \quad \forall p \in \nat : \left| \s_{m+p} - \s_m \right| < \epsilon$$ 
\end{st}
\begin{st}Необходимое условие сходимости.\\
$$ \mbox{Если ряд } \sum_{k=1}^{\infty} a_k \mbox{ сходится, то } a_k \xrightarrow{k\to\infty}0$$
\end{st}
\begin{deff}
Если ряд сходится и $\s$ --- его сумма, то $r_n = \s - \s_n$ называется остатком ряда с номером n.%
\end{deff}
\begin{deff}
Ряд $\ssum a_k $ называют абсолютно сходящимся, если сходится ряд $\ssum |a_k| $.
\end{deff}
\begin{thm}
Если ряд $\ssum a_k$ абсолюно сходится, то он сходится.
\end{thm}
\begin{proof}
По критерию Коши, $ \forall{\epsilon\!>\! 0} \quad \exists N \quad \forall m \! >\! N \quad \forall p \in \nat : \left| \s_{m+p} - \s_m \right| < \epsilon$
Так как $\left| \s_{m+p} - \s_m \right| \le \sum_{k=m+1}^{m+p}|a_k|$, то выполняется критерий Коши для исходного ряда, и он сходится.
\end{proof}
\begin{deff}
Если ряд сходится, но не сходится абсолютно, то его называют сходящимся условно.
\end{deff}
\section*{Свойства}
\begin{enumerate}
\item
Если ряд $\ssum$ сходится (сходится абсолютно) и $\s$ --- его сумма, то для любого числа $\alpha$ ряд $\ssum \alpha a_k$ сходится.
\begin{proof}
$$\ssumn \alpha a_k = \alpha \ssumn a_k = \alpha \s_n$$
Если сущестует предел частичных сумм исходного ряда равный $\s$, то $\s_\alpha = \alpha \s$
\end{proof}
\item
Если ряды $\ssum a_k$ и $\ssum b_k$ сходятся (абсолютно сходятся)
и $\s_a$ и $\s_b$ --- их суммы, то ряд $\ssum(a_k+b_k)$ сходится (сходится абсолютно) и его сумма --- $\s_a+\s_b$
\begin{proof}
$$\ssumn(a_k+b_k) = \ssumn a_k + \ssum b_k \xrightarrow{n\to\infty} \s_a + \s_b$$
Так как $$ \ssumn|a_k+b_k|\le \ssumn|a_k| + \ssumn|b_k| \le\ssum|a_k| + \ssum|b_k| \mbox{, то } \ssumn|a_k+b_k|$$ --- монотонная ограниченная последовательность и, следовательно, сходится.
\end{proof}
\item
Если ряд $\ssum a_k$ сходится (абсолютно сходится) и $\s$ его сумма, $n_k$ --- строго возрастающая последовательность натуральных чисел, то ряд $\ssum \left(\sum_{j=n_{k-1}+1}^{n_k} a_j\right)$, где $n_0=0$, сходится (сходится абсолютно) и $\s$ --- его сумма.
\begin{proof}
Последовательность частичных сумм сгруппированного ряда --- это подпоследовательность $ S_{n_k} $ последовательности частичных сумм начального ряда.
Последовательность:
$$ \sum_{k=1}^N \left| \sum_{j=n_{k-1}}^{n_N} a_j \right| \le \sum_{j=1}^{n_N}|a_j| $$
ограничена, тогда первая сумма --- монотонная ограниченная последовательность, которая сходится.
\end{proof}
\item
Если члены ряда $a_k \xrightarrow{k \to \infty} 0$, $n_k$ --- строго возрастающая последовательность натуральных чисел и $\sup_k (n_k - n_{k-1}) <\infty)$, 
сгруппированный ряд $\ssum \Big(\sum\limits_{j=n_{k-1}+1}^{n_k} a_j\Big)$, где
$ n_0 = 0 $, сходится $\s$ --- его, сумма то начальный ряд также сходится и $\s$ --- его сумма.
\begin{proof}
Для любого $ m \in \nat$ найдем такое натуральное $r$, что $n_{r-1} < m \le n_r$. Тогда: 
$$ \left| \s_{n_r} - \s_m \right| \le \sum_{k=m+1}^{n_r} |a_k| \le \sum_{k=n_{r-1}+1}^{n_r} |a_k| \le \sum_{k=n_r -l}^{n_r} |a_k|, $$
где $sup_k(n_k-n_{k-1}) \le l$.
Последняя сумма --- конечная сумма $\stackrel{=}o\!(1)$, значит, $\stackrel{=}O\!(1)$.
Следовательно, если $\s_{n_r}\xrightarrow{r\to\infty}\s$, то $\s_{m} э\xrightarrow{m\to\infty}\s$
\end{proof}
\end{enumerate}

\chapter{}
\section*{Признаки сходимости рядов неотрицательных числел}
TODO: Вставить замечания
\begin{enumerate}
\item
Если дан ряд $\ssum a_k$, то ряд сходится тогда и только тогда, когда последовательность частичных сумм $\s_n$ ограничена.
\item
(Признак сравнения) Если даны  ряды $\ssum a_k $ и $\ssum b_k , 0 \le a_k \le b_k, k>\mathbf K$, то из сходимости второго ряда следует сходимость первого, а из расходимости первого расходимость второго.
\begin{proof}
$$ \ssumn a_k \le \sum_{k=1}^{K} + \sum_{k=K+1} a_k \le \sum_{k=1}^{K} + \sum_{k=K+1} b_k \le  \sum_{k=1}^{K} + \sum_{k=1} b_k \le$$
Из сходимости второго ряда следует ограниченность частиных сумм первого, а значит и сходимость первого.\\
Если $n>K, p \ge 0$, то $\sum_{k=n+1}^{n+p}a_k \le \sum_{k=n+1}^{n+p} b_k$, поэтому из выполнения критерия Коши для второго ряда следует выполнение критерия Коши для превого. 
\end{proof}
\item
(Сравннения) Пусть $\ssum a_k $ и $\ssum b_k$ -- числовые ряды с неотрицательными членами. Если $$0 < \alpha \le \frac{a_k}{b_k} \le \beta < \infty, \forall k>K,$$
то числовые ряды одновременно сходятся или одновременно расходятся.
\begin{proof}
$0 \le a_k \le \beta b_k$, поэтому, если ряд $\ssum b_k$ сходится, то сходится $\ssum a_k$.\\
Так как $0 \le b_k \le \frac{a_k}{\alpha}$,
То если ряд $\ssum a_k$ сходится, то сходится и второй.
\end{proof}
\item
(Сравнения) Пусть $\ssum a_k $ и $\ssum b_k$ -- числовые ряды со строго положительными членами. Если
$$ \frac{a_{k+1}}{a_k}\le \frac{b_{k+1}}{b_k}, \forall k \ge K,$$
то из сходимости ряда $\ssum b_k$ следует сходимость ряда $\ssum a_k$
\begin{proof}
$$ \prod_{k+K}^{n-1}\frac{a_{k+1}}{a_k} \le \prod_{k+K}^{n-1}\frac{b_{k+1}}{b_k}\text{, т.е. } \frac{a_{n}}{a_K} \le \frac{b_{n}}{b_K}, n>K$$
Значит, $a_n \le \frac{a_K}{b_K}b_n, n>K$, поэтому из сходимости ряда $\ssum b_k$ следует сходимость ряда $\ssum a_k$
\end{proof}
\item
(Д`Алабера) Пусть $\ssum a_k$ -- ряд с неотрицательными членами. Если
$$\frac{a_{k+1}}{a_k} \le q < 1 \text{при} n\ge K$$
то ряд сходится. Если
$$ \frac{a_{k+!}}{a_k} \text{при} n\ge K$$
То члены ряда не стремятся к нулю, и ряд расходится
\begin{proof}
Возьмём $b_k = q^k$ -- геометрическую прогрессию, ряд $\ssum q^k$ сходится.
Далее используем признак сравнение. Другой случай очевиден.
\end{proof}
\item
(Коши) Пусть $\ssum a_k$ -- ряд с неотрицательными членами. Если $ \sqrt[n]{a_k}\le q < 1$ при $k \ge K $,
то ряд сходится, а если $ \sqrt[k]{a_k}\ge 1$ для бесконечного числа номеров, то члены ряда не стремятся к нулю и ряд расходится.
\begin{proof}
Если  $\sqrt[n]{a_k}\le q < 1$, то повторому признаку ряд расходится. 
\end{proof}
\item
(Интегральный Маклорена-Коши) Пусть $f(x)$ -- неотрицательная невозрастающая функция на $ [1,+\infty] $. Тогда
$$ 0 \le \ssumn f(k) - \int\limits_1^{n+1}f(x)dx\le f(1)$$
Ряд и интеграл одновременно сходятся или одновременно расходятся.
\begin{proof}
$$ 0\le \ssumn \left( f(k) - \int\limits_k^{k+1}f(x)dx \right) = \ssumn f(k) - \int\limits_1^{n+1}f(x)dx = $$
$$ = f(1)+\sum_{k=2}^n \left( f(k) - \int\limits_{k-1}^k f(x)dx \right) - \int\limits_n^{n+1} f(x)dx \le f(1)$$
Тогда частичные интеграллы и суммы ограниченны одновременно.
\end{proof}
\item
(Признак Куммера) Пусть $\ssum a_k$ -- ряд со строго положительными членами,
$b_k$ -- последовательность строго положительных чисел, $$ v_k = \frac{a_k}{a_{k+1}}b_k - b_{k+1}$$
Если $v_k \ge l . 0$ при $k \ge K$, то ряд сходится. Если $v_l \le 0$ при $k \ge K$ и ряд $\ssum (b_k)^{-1}$ расходится, то ряд $\ssum a_k$ также расходится.
\begin{proof}
$$v_k = \frac{a_k}{a_{k+1}} b_k - b_{k+1} \ge l \ge l > 0, k \ge K$$
$$a_k b_k-a_{k+1}b_{k+1} \ge l a_{k+1}, k\ge K$$
$$\sum_{k=K}^{n-1} \big( a_k b_k-a_{k+1}b_{k+1} \big) = a_Kb_K - a_nb_n \ge l \sum_{k=K}^{n-1} a_{k+1} = l \sum_{j=K+1}^{n}a_{j}$$
Так как $a_Kb_K \ge l\sum_{j=K+1}{n}$, то частичные суммы ряда $\sum_{j=K+1}{\infty}$ ограничены и, значит, ряд $\ssum a_k$ сходится.
Итак, $$v_k = \frac{a_k}{a_{k+1}}b_k - b_{k+1} \le 0 \text{ при } k\ge K,$$
т.е. $a_k b_k - a_{k+!}b_{k+1}\le 0 k \ge$. Значит, $$ \sum_{k=K}^{n-1} \big( a_k b_k-a_{k+1}b_{k+1} \big) = a_K b_K - a_n b_n \le 0, n>K$$
отсюда: $a_N \ge a_Kb_K - (b_n)^{-1}$. Из расходимости ряда $\ssum b_k$ следует расходимость ряда $\ssum a_k$
\end{proof}
\item
(Признак Раабе) Пусть $\ssum a_k$ -- ряд со строго положительными членами.
Если $$k \left( \frac{a_k}{a_{k+1}} -1 \right) > q > 1, k \ge K,$$
то ряд сходится.
Если $$k \left( \frac{a_k}{a_{k+1}} -1 \right) \le 1, k \ge K,$$
то ряд расходится.
\begin{proof}
Возьмем $b_k =k$, ряд $\ssum (b_k)^{-1}$ расходится.
$$v_k = \frac{a_k}{a_{k+1}}k - (k+1) = k (\frac{a_k}{a_{k+1}}) -1 $$
И пользуемся признаком Куммера.
\end{proof}
\item
(Признак Гаусса) Пусть 
$$(\forall k \in \nat) a_k>0, \frac{a_k}{a_{k+1}} = \alpha + \frac{\beta}{k}+\frac{\gamma_k}{k^{1+\epsilon}}, $$
где $\alpha, \beta, \epsilon \in \real, \epsilon > 0, \gamma_k$ -- ограниченная числова последовательность.
Тогда при $\alpha > 1$ или $\alpha = 1, \beta > 1$, ряд $\ssum a_k$ сходится, а  при $\alpha < 1$ или $\alpha = 1, \beta \le 1$ -- расходится.
\begin{proof}
$$\lim_{k \to \infty} \frac{a_k}{a_{k+1}} = \alpha, \lim_{k \to \infty} \frac{a_{k+1}}{a_k} = \frac{1}{\alpha} (\alpha \not=0)$$
Тогда по признаку Даламбера ряд сходится при $\alpha > 1$ и расходится при $\alpha < 1$\\
При $\alpha = 1$\\
$$\lim{k \to \infty} k ( \frac{a_k}{a_{k+1}} -1 ) = \beta $$
тогда по признаку Раабе ряд сходится при $\beta > 1$ и расходится при $\b < 1$\\
При $\alpha =1, \beta =1$ воспользуемся признаком Куммера с $b_k = k ln k, k\ge 2.$\\
Ряд обратных к $b$ расходится по интегральному признаку, так как расходится соответствующий интеграл
$$v_k = \frac{a_k}{a_{k+1}}b_k - b_{k+1} =$$
$$= \left(1 + \frac{1}{k} + \frac{\gamma_k}{k^{1+\epsilon}} \right) k lnk - (k+1)ln(k+1) =$$
$$=(k+1)\ln k + \frac{\gamma_k}{k^{\epsilon}}ln k - (k+1)\ln(k+1) = \frac{\gamma_k \ln k}{k^{\epsilon}} - \ln(\frac{k+1}{k})^{k+1}$$
$$\frac{\gamma_k \ln k}{k^{\epsilon}} \xrightarrow{k \to \infty} 0$$
$$\ln(\frac{k+1}{k})^{k+1} \xrightarrow{k \to \infty} 1$$
$$\frac{\gamma_k \ln k}{k^{\epsilon}} - \ln(\frac{k+1}{k})^{k+1} \to -1 $$
\end{proof}
\end{enumerate}
\begin{thm}
Если выполнено условие сходимости Д'Аламбера, то выполнено условие сходимости ряда Коши.
\end{thm}
\begin{proof}
$$a_n = a_K \prod_{k=K}^{n-1} \frac{a_{k+1}}{a_k} \le a_K q^{n-K}, \sqrt[n]{a_n} \le \sqrt[n]{a_K}q^{1-\frac{K}{n}} \xrightarrow{n \to \infty} 1*q < 1$$
Если взять $p, q<p<1$, то, начиная с некоторого номера $\sqrt[n]{a_n}<p<1$
\end{proof}
\chapter[Не только положительные ряды]{Ряды с членами разных знаков или с членами --- комплексными числами.}

\begin{thm}
\textbf{(Признак Лейбница)} Пусть $\ssum a_k$ --- ряд со знакочередующимся членами,
которыепо модулю монотонно стремятся к нулю.
Тогда ряд сходится и остаток $|r_n| \ge |a_{n+1}| \ge |a_n|$.
\end{thm}
\begin{proof}
Пусть $a_1 > 0$. Тогда $\s_{2n} = \ssumn (a_{2k-1}+a_{2k})$ -- неубывающая последовательность, $\s_{2n+1} = a_1 + \ssumn(a_{2k} + a_{2k+1})$ --- невозрастающая последовательность,
$\s_{2n} \le S_{2n+1} = a_1+\ssumn( a_{2k} + a_{2k+1}) \le 0$
--- невозрастающая последовательность, $\s_{2n} \le \s_{2n+1}$.

Значит, $\s_{2n}$ --- неубывающая, ограниченная сверху последовательность.
$\s_{2n+1}$ --- невозрастающая, ограниченная снизу последовательность.
Они сходятся и, так как $\s_{2n+1}-\s_{2n}=a_{2n+1}= \overline{o}(1)$,
то имеют бощий предел $\s$.
\\
$$r_{2n+1} = \sum_{k=2n+2}^{\infty} a_k =
\underbrace{a_{2n+2}}_{\le 0} +  \sum_{k=n+2}^{\infty}\underbrace{\big(a_{2k-1} + a_{2k}\big)}_{\ge 0} =  \sum_{k=n+1}^{\infty} \underbrace{\big(a_{2k}+a_{2k+1} \big)}_{\le 0}$$
Следовательно,
$$a_{2n+2} \le r_{2n+1} \le0, |r_{2n+1}| \le |a_{2n+2}| \le |a_{2n+2}|$$
$$r_{2n} = \sum_{k=2n+1}^{\infty} a_k = \sum_{k=n}^{\infty} \underbrace{\big( a_{2k+1} + a_{2k+2} \big) }_{\ge 0} =
\underbrace{a_{2n+1}}_{\ge 0} + \sum_{k=n+1}^{\infty} \underbrace{ \big(a_{2k-1} + a_{2k}\big)}_{\le 0}$$
Следовательно $0 \le r_{2n} \le a_{2n+1} \le |a_{2n}|$
\end{proof}
\newpage
\subsection*{Преобразование Абеля}

$$\sum_{k=m}^n u_k v_k = \sum_{s=m-1}^{n-1} U_k \big( v_k - v_{k+1} \big) + U_nv_n-U_{m-1}v_{m-1} =$$
$$=\sum_{k=m}^n U_k(v_k-v_{k+1}) +  U_nv_n - U_{m-1}u_m,$$
$$ \text{где } U_n=\ssumn u_k, U_0 = 0, v_0 =0$$
\begin{proof}
\begin{multline*}
\sum_{k=m}^n u_k v_k = \sum_{k=m}^n \big(U_k-U_{k-1} \big) v_k = \sum_{k=m}^n U_kv_k - \sum_{k=m}^nU_{k-1}v_k=\\
=\sum_{k=m}^{n}U_k v_k - \sum_{k=m-1}^{n-1} U_kv_{k+1} = \sum_{k=m-1}^{n-1}U_k(v_k-v_{k+1})+V_nv_n-U_{m-1}v_{m-1}=\\
=\sum_{k=m}^{n-1} U_k (v_k - v_{k+1}) + U_nv_n - U_{m-1}v_m
\end{multline*}
\end{proof}
\subsection*{Последовательность ограниченной вариации.}
\begin{deff}
Последовательность $\{v_k\}_{k=1}^{\infty}$ называется последовательностью ограниченной вариации, если сходится ряд модулей разниц между соседними членами
\end{deff}
\begin{thm}
Последовательность действительных чисел  является последовательностью ограниенной вариации тогда и только тогда, когда её можно представить как разность двух неубывающих(невозрастающих) сходящихся последовательностей
\end{thm}
\begin{proof}
(Достаточность)
Монотонная сходящаяся последовательность $v_k$ является последовательностью ограниченнной вариации.
Действительно, если $v_k$ --- невозрастающая последовательность, то $\ssumn |v_k-v_{k+1}| = \ssumn(v_k-v_k+1)= v_1-v_{n+1}$
--- имеет предел при $n \to  \infty$, следовательно, последовательность имеет ограниченную вариацию.

Сумма, разность последовательностей ограниченной вариации являются последовательностями ограниченной вариации. Для любого числа $\alpha$ и VB-последовательности $v_k$ \; $\alpha v_k$ ограниченна.

(Необходимость) Последовательность $S_n = \sum_{k=1}^{n-1} |v_k-v_{k+1}|$ --- неубывающая, $G_n = \sum_{k=1}^{n-1} \big |v_k -v_k+1|+(v_k-v_{k+1})$ --- неубывающая последовательность, $v_n = v_1 - \sum _{k=1}^{n-1}(v_k - v_{k+1})= S_n + v_1 - G_n=S_n -(-v_1 + G_n)$
Домножением на минус единицу можно получить рвзность двух невозрастающей последовательностей.
\end{proof}
\begin{thm}
Если $v_n$ --- последовательность ограниченной вариации, то она сходится
\end{thm}
\begin{proof}
$ v_n = v_1 + \sum_{k=1}^{n-1}(v_{k+1} - v_k)$, ряд $\sum_{k=1}^{n-1}( v_{k+1} - v_k) $ сходится абсолютно.
\end{proof}

\end{document}
