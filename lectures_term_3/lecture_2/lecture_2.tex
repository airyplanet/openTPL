\chapter{}
\section*{Признаки сходимости рядов неотрицательных числел}
TODO: Вставить замечания
\begin{enumerate}
\item
Если дан ряд $\ssum a_k$, то ряд сходится тогда и только тогда, когда последовательность частичных сумм $\s_n$ ограничена.
\item
(Признак сравнения) Если даны  ряды $\ssum a_k $ и $\ssum b_k , 0 \le a_k \le b_k, k>\mathbf K$, то из сходимости второго ряда следует сходимость первого, а из расходимости первого расходимость второго.
\begin{proof}
$$ \ssumn a_k \le \sum_{k=1}^{K} + \sum_{k=K+1} a_k \le \sum_{k=1}^{K} + \sum_{k=K+1} b_k \le  \sum_{k=1}^{K} + \sum_{k=1} b_k \le$$
Из сходимости второго ряда следует ограниченность частиных сумм первого, а значит и сходимость первого.\\
Если $n>K, p \ge 0$, то $\sum_{k=n+1}^{n+p}a_k \le \sum_{k=n+1}^{n+p} b_k$, поэтому из выполнения критерия Коши для второго ряда следует выполнение критерия Коши для превого. 
\end{proof}
\item
(Сравннения) Пусть $\ssum a_k $ и $\ssum b_k$ -- числовые ряды с неотрицательными членами. Если $$0 < \alpha \le \frac{a_k}{b_k} \le \beta < \infty, \forall k>K,$$
то числовые ряды одновременно сходятся или одновременно расходятся.
\begin{proof}
$0 \le a_k \le \beta b_k$, поэтому, если ряд $\ssum b_k$ сходится, то сходится $\ssum a_k$.\\
Так как $0 \le b_k \le \frac{a_k}{\alpha}$,
То если ряд $\ssum a_k$ сходится, то сходится и второй.
\end{proof}
\item
(Сравнения) Пусть $\ssum a_k $ и $\ssum b_k$ -- числовые ряды со строго положительными членами. Если
$$ \frac{a_{k+1}}{a_k}\le \frac{b_{k+1}}{b_k}, \forall k \ge K,$$
то из сходимости ряда $\ssum b_k$ следует сходимость ряда $\ssum a_k$
\begin{proof}
$$ \prod_{k+K}^{n-1}\frac{a_{k+1}}{a_k} \le \prod_{k+K}^{n-1}\frac{b_{k+1}}{b_k}\text{, т.е. } \frac{a_{n}}{a_K} \le \frac{b_{n}}{b_K}, n>K$$
Значит, $a_n \le \frac{a_K}{b_K}b_n, n>K$, поэтому из сходимости ряда $\ssum b_k$ следует сходимость ряда $\ssum a_k$
\end{proof}
\item
(Д`Алабера) Пусть $\ssum a_k$ -- ряд с неотрицательными членами. Если
$$\frac{a_{k+1}}{a_k} \le q < 1 \text{при} n\ge K$$
то ряд сходится. Если
$$ \frac{a_{k+!}}{a_k} \text{при} n\ge K$$
То члены ряда не стремятся к нулю, и ряд расходится
\begin{proof}
Возьмём $b_k = q^k$ -- геометрическую прогрессию, ряд $\ssum q^k$ сходится.
Далее используем признак сравнение. Другой случай очевиден.
\end{proof}
\item
(Коши) Пусть $\ssum a_k$ -- ряд с неотрицательными членами. Если $ \sqrt[n]{a_k}\le q < 1$ при $k \ge K $,
то ряд сходится, а если $ \sqrt[k]{a_k}\ge 1$ для бесконечного числа номеров, то члены ряда не стремятся к нулю и ряд расходится.
\begin{proof}
Если  $\sqrt[n]{a_k}\le q < 1$, то повторому признаку ряд расходится. 
\end{proof}
\item
(Интегральный Маклорена-Коши) Пусть $f(x)$ -- неотрицательная невозрастающая функция на $ [1,+\infty] $. Тогда
$$ 0 \le \ssumn f(k) - \int\limits_1^{n+1}f(x)dx\le f(1)$$
Ряд и интеграл одновременно сходятся или одновременно расходятся.
\begin{proof}
$$ 0\le \ssumn \left( f(k) - \int\limits_k^{k+1}f(x)dx \right) = \ssumn f(k) - \int\limits_1^{n+1}f(x)dx = $$
$$ = f(1)+\sum_{k=2}^n \left( f(k) - \int\limits_{k-1}^k f(x)dx \right) - \int\limits_n^{n+1} f(x)dx \le f(1)$$
Тогда частичные интеграллы и суммы ограниченны одновременно.
\end{proof}
\item
(Признак Куммера) Пусть $\ssum a_k$ -- ряд со строго положительными членами,
$b_k$ -- последовательность строго положительных чисел, $$ v_k = \frac{a_k}{a_{k+1}}b_k - b_{k+1}$$
Если $v_k \ge l . 0$ при $k \ge K$, то ряд сходится. Если $v_l \le 0$ при $k \ge K$ и ряд $\ssum (b_k)^{-1}$ расходится, то ряд $\ssum a_k$ также расходится.
\begin{proof}
$$v_k = \frac{a_k}{a_{k+1}} b_k - b_{k+1} \ge l \ge l > 0, k \ge K$$
$$a_k b_k-a_{k+1}b_{k+1} \ge l a_{k+1}, k\ge K$$
$$\sum_{k=K}^{n-1} \big( a_k b_k-a_{k+1}b_{k+1} \big) = a_Kb_K - a_nb_n \ge l \sum_{k=K}^{n-1} a_{k+1} = l \sum_{j=K+1}^{n}a_{j}$$
Так как $a_Kb_K \ge l\sum_{j=K+1}{n}$, то частичные суммы ряда $\sum_{j=K+1}{\infty}$ ограничены и, значит, ряд $\ssum a_k$ сходится.
Итак, $$v_k = \frac{a_k}{a_{k+1}}b_k - b_{k+1} \le 0 \text{ при } k\ge K,$$
т.е. $a_k b_k - a_{k+!}b_{k+1}\le 0 k \ge$. Значит, $$ \sum_{k=K}^{n-1} \big( a_k b_k-a_{k+1}b_{k+1} \big) = a_K b_K - a_n b_n \le 0, n>K$$
отсюда: $a_N \ge a_Kb_K - (b_n)^{-1}$. Из расходимости ряда $\ssum b_k$ следует расходимость ряда $\ssum a_k$
\end{proof}
\item
(Признак Раабе) Пусть $\ssum a_k$ -- ряд со строго положительными членами.
Если $$k \left( \frac{a_k}{a_{k+1}} -1 \right) > q > 1, k \ge K,$$
то ряд сходится.
Если $$k \left( \frac{a_k}{a_{k+1}} -1 \right) \le 1, k \ge K,$$
то ряд расходится.
\begin{proof}
Возьмем $b_k =k$, ряд $\ssum (b_k)^{-1}$ расходится.
$$v_k = \frac{a_k}{a_{k+1}}k - (k+1) = k (\frac{a_k}{a_{k+1}}) -1 $$
И пользуемся признаком Куммера.
\end{proof}
\item
(Признак Гаусса) Пусть 
$$(\forall k \in \nat) a_k>0, \frac{a_k}{a_{k+1}} = \alpha + \frac{\beta}{k}+\frac{\gamma_k}{k^{1+\epsilon}}, $$
где $\alpha, \beta, \epsilon \in \real, \epsilon > 0, \gamma_k$ -- ограниченная числова последовательность.
Тогда при $\alpha > 1$ или $\alpha = 1, \beta > 1$, ряд $\ssum a_k$ сходится, а  при $\alpha < 1$ или $\alpha = 1, \beta \le 1$ -- расходится.
\begin{proof}
$$\lim_{k \to \infty} \frac{a_k}{a_{k+1}} = \alpha, \lim_{k \to \infty} \frac{a_{k+1}}{a_k} = \frac{1}{\alpha} (\alpha \not=0)$$
Тогда по признаку Даламбера ряд сходится при $\alpha > 1$ и расходится при $\alpha < 1$\\
При $\alpha = 1$\\
$$\lim{k \to \infty} k ( \frac{a_k}{a_{k+1}} -1 ) = \beta $$
тогда по признаку Раабе ряд сходится при $\beta > 1$ и расходится при $\b < 1$\\
При $\alpha =1, \beta =1$ воспользуемся признаком Куммера с $b_k = k ln k, k\ge 2.$\\
Ряд обратных к $b$ расходится по интегральному признаку, так как расходится соответствующий интеграл
$$v_k = \frac{a_k}{a_{k+1}}b_k - b_{k+1} = \left(1 + \frac{1}{k} + \frac{\gamma_k}{k^{1+\epsilon}} \right) k lnk - (k+1)ln(k+1) = 
(k+1)lnk + \frac{\gamma_k}{k^{\epsilon}}ln k - (k+1)ln(k+1) =$$
$$ = \frac{\gamma_k lnk}{k^{epsilon}} - ln(\frac{k+1}{k})^{k+1}$$
$$\frac{\gamma_k lnk}{k^{epsilon}} \xrightarrow{k \to \infty} 0$$
$$ln(\frac{k+1}{k})^{k+1} \xrightarrow{k \to \infty} 1$$
$$\frac{\gamma_k lnk}{k^{epsilon}} - ln(\frac{k+1}{k})^{k+1} \to \infty $$
\end{proof}
\end{enumerate}
\begin{thm}
Если выполнено условие сходимости Д'Аламбера, то выполнено условие сходимости ряда Коши.
\end{thm}
\begin{proof}
$$a_n = a_K \prod_{k=K}^{n-1} \frac{a_{k+1}}{a_k} \le a_K q^{n-K}, \sqrt[n]{a_n} \le \sqrt[n]{a_K}q^{1-\frac{K}{n}} \xrightarrow{n \to \infty} 1*q < 1$$
Если взять $p, q<p<1$, то, начиная с некоторого номера $\sqrt[n]{a_n}<p<1$
\end{proof}